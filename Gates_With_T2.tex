% ##OpenQuantumSystems
\documentclass[10pt,a4paper, english]{scrartcl}
\usepackage[utf8]{inputenc}
\usepackage{amsmath}
\usepackage{amsfonts}
\usepackage{amssymb}
\usepackage{babel}
\usepackage[cm]{fullpage}
\usepackage{float}
\usepackage{graphicx}
\usepackage{helvet}
\usepackage{hyperref}
\usepackage{mathtools}
\usepackage{nicefrac}
\usepackage{tikz}
\usepackage{pgfplots}
\usepackage{placeins}
\usepackage{verbatim}
\usepackage{xcolor}
\definecolor{gr}{gray}{0.9}
\renewcommand{\familydefault}{\sfdefault}
\title{Effect of Decoherence During Gates}
\author{Ben Criger}
\date{\today}
%    Q-circuit version 2
%    Copyright (C) 2004  Steve Flammia & Bryan Eastin
%    Last modified on: 9/16/2011
%
%    This program is free software; you can redistribute it and/or modify
%    it under the terms of the GNU General Public License as published by
%    the Free Software Foundation; either version 2 of the License, or
%    (at your option) any later version.
%
%    This program is distributed in the hope that it will be useful,
%    but WITHOUT ANY WARRANTY; without even the implied warranty of
%    MERCHANTABILITY or FITNESS FOR A PARTICULAR PURPOSE.  See the
%    GNU General Public License for more details.
%
%    You should have received a copy of the GNU General Public License
%    along with this program; if not, write to the Free Software
%    Foundation, Inc., 59 Temple Place, Suite 330, Boston, MA  02111-1307  USA

% Thanks to the Xy-pic guys, Kristoffer H Rose, Ross Moore, and Daniel Müllner,
% for their help in making Qcircuit work with Xy-pic version 3.8.  
% Thanks also to Dave Clader, Andrew Childs, Rafael Possignolo, Tyson Williams,
% Sergio Boixo, Cris Moore, Jonas Anderson, and Stephan Mertens for helping us test 
% and/or develop the new version.

\usepackage{xy}
\xyoption{matrix}
\xyoption{frame}
\xyoption{arrow}
\xyoption{arc}

\usepackage{ifpdf}
\ifpdf
\else
\PackageWarningNoLine{Qcircuit}{Qcircuit is loading in Postscript mode.  The Xy-pic options ps and dvips will be loaded.  If you wish to use other Postscript drivers for Xy-pic, you must modify the code in Qcircuit.tex}
%    The following options load the drivers most commonly required to
%    get proper Postscript output from Xy-pic.  Should these fail to work,
%    try replacing the following two lines with some of the other options
%    given in the Xy-pic reference manual.
\xyoption{ps}
\xyoption{dvips}
\fi

% The following resets Xy-pic matrix alignment to the pre-3.8 default, as
% required by Qcircuit.
\entrymodifiers={!C\entrybox}

\newcommand{\bra}[1]{{\left\langle{#1}\right\vert}}
\newcommand{\ket}[1]{{\left\vert{#1}\right\rangle}}
    % Defines Dirac notation. %7/5/07 added extra braces so that the commands will work in subscripts.
\newcommand{\qw}[1][-1]{\ar @{-} [0,#1]}
    % Defines a wire that connects horizontally.  By default it connects to the object on the left of the current object.
    % WARNING: Wire commands must appear after the gate in any given entry.
\newcommand{\qwx}[1][-1]{\ar @{-} [#1,0]}
    % Defines a wire that connects vertically.  By default it connects to the object above the current object.
    % WARNING: Wire commands must appear after the gate in any given entry.
\newcommand{\cw}[1][-1]{\ar @{=} [0,#1]}
    % Defines a classical wire that connects horizontally.  By default it connects to the object on the left of the current object.
    % WARNING: Wire commands must appear after the gate in any given entry.
\newcommand{\cwx}[1][-1]{\ar @{=} [#1,0]}
    % Defines a classical wire that connects vertically.  By default it connects to the object above the current object.
    % WARNING: Wire commands must appear after the gate in any given entry.
\newcommand{\gate}[1]{*+<.6em>{#1} \POS ="i","i"+UR;"i"+UL **\dir{-};"i"+DL **\dir{-};"i"+DR **\dir{-};"i"+UR **\dir{-},"i" \qw}
    % Boxes the argument, making a gate.
\newcommand{\meter}{*=<1.8em,1.4em>{\xy ="j","j"-<.778em,.322em>;{"j"+<.778em,-.322em> \ellipse ur,_{}},"j"-<0em,.4em>;p+<.5em,.9em> **\dir{-},"j"+<2.2em,2.2em>*{},"j"-<2.2em,2.2em>*{} \endxy} \POS ="i","i"+UR;"i"+UL **\dir{-};"i"+DL **\dir{-};"i"+DR **\dir{-};"i"+UR **\dir{-},"i" \qw}
    % Inserts a measurement meter.
    % In case you're wondering, the constants .778em and .322em specify
    % one quarter of a circle with radius 1.1em.
    % The points added at + and - <2.2em,2.2em> are there to strech the
    % canvas, ensuring that the size is unaffected by erratic spacing issues
    % with the arc.
\newcommand{\measure}[1]{*+[F-:<.9em>]{#1} \qw}
    % Inserts a measurement bubble with user defined text.
\newcommand{\measuretab}[1]{*{\xy*+<.6em>{#1}="e";"e"+UL;"e"+UR **\dir{-};"e"+DR **\dir{-};"e"+DL **\dir{-};"e"+LC-<.5em,0em> **\dir{-};"e"+UL **\dir{-} \endxy} \qw}
    % Inserts a measurement tab with user defined text.
\newcommand{\measureD}[1]{*{\xy*+=<0em,.1em>{#1}="e";"e"+UR+<0em,.25em>;"e"+UL+<-.5em,.25em> **\dir{-};"e"+DL+<-.5em,-.25em> **\dir{-};"e"+DR+<0em,-.25em> **\dir{-};{"e"+UR+<0em,.25em>\ellipse^{}};"e"+C:,+(0,1)*{} \endxy} \qw}
    % Inserts a D-shaped measurement gate with user defined text.
\newcommand{\multimeasure}[2]{*+<1em,.9em>{\hphantom{#2}} \qw \POS[0,0].[#1,0];p !C *{#2},p \drop\frm<.9em>{-}}
    % Draws a multiple qubit measurement bubble starting at the current position and spanning #1 additional gates below.
    % #2 gives the label for the gate.
    % You must use an argument of the same width as #2 in \ghost for the wires to connect properly on the lower lines.
\newcommand{\multimeasureD}[2]{*+<1em,.9em>{\hphantom{#2}} \POS [0,0]="i",[0,0].[#1,0]="e",!C *{#2},"e"+UR-<.8em,0em>;"e"+UL **\dir{-};"e"+DL **\dir{-};"e"+DR+<-.8em,0em> **\dir{-};{"e"+DR+<0em,.8em>\ellipse^{}};"e"+UR+<0em,-.8em> **\dir{-};{"e"+UR-<.8em,0em>\ellipse^{}},"i" \qw}
    % Draws a multiple qubit D-shaped measurement gate starting at the current position and spanning #1 additional gates below.
    % #2 gives the label for the gate.
    % You must use an argument of the same width as #2 in \ghost for the wires to connect properly on the lower lines.
\newcommand{\control}{*!<0em,.025em>-=-<.2em>{\bullet}}
    % Inserts an unconnected control.
\newcommand{\controlo}{*+<.01em>{\xy -<.095em>*\xycircle<.19em>{} \endxy}}
    % Inserts a unconnected control-on-0.
\newcommand{\ctrl}[1]{\control \qwx[#1] \qw}
    % Inserts a control and connects it to the object #1 wires below.
\newcommand{\ctrlo}[1]{\controlo \qwx[#1] \qw}
    % Inserts a control-on-0 and connects it to the object #1 wires below.
\newcommand{\targ}{*+<.02em,.02em>{\xy ="i","i"-<.39em,0em>;"i"+<.39em,0em> **\dir{-}, "i"-<0em,.39em>;"i"+<0em,.39em> **\dir{-},"i"*\xycircle<.4em>{} \endxy} \qw}
    % Inserts a CNOT target.
\newcommand{\qswap}{*=<0em>{\times} \qw}
    % Inserts half a swap gate.
    % Must be connected to the other swap with \qwx.
\newcommand{\multigate}[2]{*+<1em,.9em>{\hphantom{#2}} \POS [0,0]="i",[0,0].[#1,0]="e",!C *{#2},"e"+UR;"e"+UL **\dir{-};"e"+DL **\dir{-};"e"+DR **\dir{-};"e"+UR **\dir{-},"i" \qw}
    % Draws a multiple qubit gate starting at the current position and spanning #1 additional gates below.
    % #2 gives the label for the gate.
    % You must use an argument of the same width as #2 in \ghost for the wires to connect properly on the lower lines.
\newcommand{\ghost}[1]{*+<1em,.9em>{\hphantom{#1}} \qw}
    % Leaves space for \multigate on wires other than the one on which \multigate appears.  Without this command wires will cross your gate.
    % #1 should match the second argument in the corresponding \multigate.
\newcommand{\push}[1]{*{#1}}
    % Inserts #1, overriding the default that causes entries to have zero size.  This command takes the place of a gate.
    % Like a gate, it must precede any wire commands.
    % \push is useful for forcing columns apart.
    % NOTE: It might be useful to know that a gate is about 1.3 times the height of its contents.  I.e. \gate{M} is 1.3em tall.
    % WARNING: \push must appear before any wire commands and may not appear in an entry with a gate or label.
\newcommand{\gategroup}[6]{\POS"#1,#2"."#3,#2"."#1,#4"."#3,#4"!C*+<#5>\frm{#6}}
    % Constructs a box or bracket enclosing the square block spanning rows #1-#3 and columns=#2-#4.
    % The block is given a margin #5/2, so #5 should be a valid length.
    % #6 can take the following arguments -- or . or _\} or ^\} or \{ or \} or _) or ^) or ( or ) where the first two options yield dashed and
    % dotted boxes respectively, and the last eight options yield bottom, top, left, and right braces of the curly or normal variety.  See the Xy-pic reference manual for more options.
    % \gategroup can appear at the end of any gate entry, but it's good form to pick either the last entry or one of the corner gates.
    % BUG: \gategroup uses the four corner gates to determine the size of the bounding box.  Other gates may stick out of that box.  See \prop.

\newcommand{\rstick}[1]{*!L!<-.5em,0em>=<0em>{#1}}
    % Centers the left side of #1 in the cell.  Intended for lining up wire labels.  Note that non-gates have default size zero.
\newcommand{\lstick}[1]{*!R!<.5em,0em>=<0em>{#1}}
    % Centers the right side of #1 in the cell.  Intended for lining up wire labels.  Note that non-gates have default size zero.
\newcommand{\ustick}[1]{*!D!<0em,-.5em>=<0em>{#1}}
    % Centers the bottom of #1 in the cell.  Intended for lining up wire labels.  Note that non-gates have default size zero.
\newcommand{\dstick}[1]{*!U!<0em,.5em>=<0em>{#1}}
    % Centers the top of #1 in the cell.  Intended for lining up wire labels.  Note that non-gates have default size zero.
\newcommand{\Qcircuit}{\xymatrix @*=<0em>}
    % Defines \Qcircuit as an \xymatrix with entries of default size 0em.
\newcommand{\link}[2]{\ar @{-} [#1,#2]}
    % Draws a wire or connecting line to the element #1 rows down and #2 columns forward.
\newcommand{\pureghost}[1]{*+<1em,.9em>{\hphantom{#1}}}
    % Same as \ghost except it omits the wire leading to the left. 

\usepackage{amsmath}
\usepackage{bbold}
\usepackage{color}
\usepackage{stmaryrd}
\usepackage{calc}
\usepackage{verbatim}
\usepackage{mathtools}
\usepackage{xspace}
\DeclarePairedDelimiter{\ceil}{\lceil}{\rceil}
\DeclarePairedDelimiter{\floor}{\lfloor}{\rfloor}
\usepackage{tikz}
\usetikzlibrary{calc}
\providecommand{\polygon}[2]{%
  let \n{len} = {2*#2*tan(360/(2*#1))} in
 ++(0,-#2) ++(\n{len}/2,0) \foreach \x in {1,...,#1} { -- ++(\x*360/#1:\n{len})}}

\DeclareMathOperator\erf{erf}
\DeclareMathOperator\erfc{erfc}

\newsavebox\CBox
\newcommand\hcancel[2][0.5pt]{%
  \ifmmode\sbox\CBox{$#2$}\else\sbox\CBox{#2}\fi%
  \makebox[0pt][l]{\usebox\CBox}%  
  \rule[0.5\ht\CBox-#1/2]{\wd\CBox}{#1}}

\providecommand{\drv}[1]{\dfrac{\partial }{\partial #1}}
\providecommand{\drf}[2]{\dfrac{\partial #1}{\partial #2}}
\providecommand{\ddrf}[3]{\dfrac{\partial^2 #1}{\partial #2 \partial #3}}

\providecommand{\tr}{\mathrm{tr}}
 
\providecommand{\ket}[1]{\left \vert #1 \right \rangle}
\providecommand{\bra}[1]{\left \langle #1 \right \vert}
\providecommand{\braket}[2]{\left \langle #1 \left \vert #2 \right. \right \rangle}
\providecommand{\angles}[1]{\left \langle #1 \right \rangle}
\providecommand{\elem}[3]{\left \langle #1 \left \vert \vphantom{#1#2#3} #2 \right \vert #3 \right \rangle}
\providecommand{\delem}[2]{\left \langle #1 \left \vert \vphantom{#1#2} #2 \right \vert #1 \right \rangle}
\providecommand{\ketbra}[2]{\ket{#1} \! \bra{#2}}
\providecommand{\proj}[1]{\ketbra{#1}{#1}}
\providecommand{\twonorm}[1]{\| #1 \|_2}
\providecommand{\abs}[1]{\left \vert #1 \right \vert}
\providecommand{\set}[1]{\left \lbrace #1 \right \rbrace}
\providecommand{\group}[1]{\left \langle #1 \right \rangle}
\providecommand{\red}[1]{\textcolor[rgb]{0.5,0,0}{#1}}
\providecommand{\blue}[1]{\textcolor[rgb]{0,0,0.5}{#1}}
\providecommand{\green}[1]{\textcolor[rgb]{0,0.5,0}{#1}}
\providecommand{\conjecture}[1]{\red{#1 (check this).}}
\providecommand{\future}[1]{\green{#1 (do this later).}}
\providecommand{\id}{\hat{\mathbb{1}}}
\providecommand{\com}[2]{\left[#1,\,#2 \right]}
\providecommand{\acom}[2]{\left \lbrace #1,\,#2 \right \rbrace}
\providecommand{\diss}[2]{\mathcal{D}\left[ #1 \right]\left( #2 \right)}
\providecommand{\meas}[2]{\mathcal{M}\left[ #1 \right]\left( #2 \right)}
\providecommand{\lindtwo}[2]{ #1 #2 #1^{\dagger} - \dfrac{1}{2} \left \lbrace #1^{\dagger} #1,\,#2 \right \rbrace }
\providecommand{\lindthree}[3]{ #1 #2 #3 - \dfrac{1}{2} \acom{#3 #1}{#2} }
\providecommand{\lindfour}[4]{ #1 #2 #3 - \dfrac{1}{2} \acom{#4}{#2} }
\providecommand{\meastwo}[2]{ #1 #2 + #2 #1^{\dagger} - \tr \left( #1 #2 + #2 #1^{\dagger} \right) #2 }
\providecommand{\tenscom}[4]{\com{#1\otimes #2}{#3 \otimes #4}=\dfrac{1}{2}\left( \com{#1}{#3} \otimes \acom{#2}{#4} + \acom{#1}{#3} \otimes \com{#2}{#4} \right)}
\providecommand{\tenscomsimple}[4]{\com{#1\otimes #2}{#3 \otimes #4} = #1 #3 \otimes \com{#2}{#4} + \com{#1}{#3} \otimes  #4 #2}
\providecommand{\tensacom}[4]{\acom{#1\otimes #2}{#3 \otimes #4}=\dfrac{1}{2}\left( \com{#1}{#3} \otimes \com{#2}{#4} + \acom{#1}{#3} \otimes \acom{#2}{#4} \right)}
\providecommand{\trace}[1]{\mathrm{tr} \left( #1 \right)}
\providecommand{\comp}{\mathop{\bigcirc}}
\providecommand{\swap}{\textsc{swap}\xspace}
\providecommand{\cnot}{\textsc{cnot}\xspace}
\providecommand{\col}{\textrm{col}}
\providecommand{\qec}[3]{\llbracket #1,\,#2,\,#3 \rrbracket}
\providecommand{\set}[1]{\left \lbrace #1 \right \rbrace}
\renewcommand{\Im}{\textrm{Im}}
\renewcommand{\Re}{\textrm{Re}}
\providecommand{\lind}{\mathcal{L}}
\providecommand{\norm}[2]{\left \vert \left \vert #2 \right \vert \right \vert_{#1}}
\providecommand{\supp}[1]{\mathrm{supp}\left( #1 \right)}
\providecommand{\given}{\, \middle \vert \,}
\providecommand{\suchthat}{\, \middle \vert \,}
\providecommand{\diag}[1]{\mathrm{diag}\left( #1 \right)}
\providecommand{\ct}{^{\dagger}}

\providecommand{\norm}[1]{\left\Vert #1 \right\Vert} %only works with amsmath

\providecommand{\ncrit}{$n_{\textrm{crit}}$\xspace}

\makeatletter
\providecommand{\pr}[2]{p\left(#1\,\middle|\,#2\right)}
\newcommand{\pushright}[1]{\ifmeasuring@#1\else\omit\hfill$\displaystyle#1$\fi\ignorespaces}
\newcommand{\pushleft}[1]{\ifmeasuring@#1\else\omit$\displaystyle#1$\hfill\fi\ignorespaces}
\makeatother
\newlength\figureheight
\newlength\figurewidth
\setlength\figureheight{7cm}
\setlength\figurewidth{12cm}

\providecommand{\cnot}{\textsc{cnot}}

\usetikzlibrary{decorations.pathreplacing,decorations.pathmorphing}

\begin{document}
\maketitle
\section{Introduction}
The purpose of this note is to determine the effect of biased (read: mostly dephasing) noise during the operation of certain gates.
I'm going to start with the $Y_{90}$ gate, implemented by a nice square pulse. 
There are two limits that can be handled succinctly, the perfectly-Markovian limit (where we can't decrease the error rate without a quantum error-correcting code), and the `unknown constant Hamiltonian' limit, where we can apply pulse sequences, reversing the effect of the unknown Hamiltonian for $\sim \nicefrac{1}{2}$ the pulse duration.
I begin with the Markovian limit, as I'm not so familiar with DD pulse sequences.
\section{Helpful Math}
Let's vectorize the density matrix, in the Pauli basis:
\begin{equation}
\rho = \frac{\id}{2} + \rho_x \sigma_x + \rho_y \sigma_y + \rho_z \sigma_z 
\end{equation}
To express the equation of motion in this basis, I calculate a few commutators and dissipators. 
\begin{flalign}
&\diss{A}{B} = \lindtwo{A}{B} \\
% &\diss{\id}{B} = \lindtwo{\id}{B} = 0 \\
&\diss{A}{\id} = \lindtwo{A}{\id} = \com{A}{A\ct}\\
&\therefore 
% &\diss{\sigma_+}{\id} = \com{\sigma_+}{\sigma_-} = -\sigma_z \quad \diss{\sigma_-}{\id} = \com{\sigma_-}{\sigma_+} = \sigma_z \quad
\diss{\sigma_z}{\id} = \com{\sigma_z}{\sigma_z} = 0 \\
&\diss{\sigma_z}{\sigma_z} = 0\\
&\diss{\sigma_z}{\sigma_{x,y}} = -2\sigma_{x,y}\\
% & \diss{\sigma_-}{\sigma_{x,y,z}} = \lindthree{\ketbra{0}{1}}{\sigma_{x,y,z}}{\ketbra{1}{0}}\\
% & \diss{\sigma_-}{\sigma_{x,y}} = -\dfrac{1}{2}\sigma_{x,y}\\
% &\diss{\sigma_-}{\sigma_z} = -\sigma_z \\ 
&\com{\sigma_y}{\sigma_x} = -2i\sigma_z \quad \com{\sigma_y}{\sigma_z} = 2i\sigma_x 
\end{flalign}
With these in hand, we can start expressing the equation of motion for a noisy $Y_{90}$ in this operator basis. 
\section{$Y_{90}$ with Markovian Noise}
A simple master equation for a $Y_{90}$, subject to dephasing is:
\begin{equation}
\dot{\rho} = -i\frac{\omega}{2} \com{\sigma_y}{\rho} + \frac{\gamma}{2} \diss{\sigma_z}{\rho},
\end{equation}
where the factors of two are included to make the matrix description look nice, as we will see in a minute. 
I express the commutator and dissipator in matrix form:
\begin{alignat}{2}
\frac{\gamma}{2} \diss{\sigma_z}{\cdot} &= \begin{bmatrix}
0 & 0 & 0 & 0 \\ 0 & -\gamma & 0 & 0 \\ 0 & 0 & -\gamma & 0 \\ 0 & 0 & 0 & 0
\end{bmatrix}
\quad 
-i \frac{\omega}{2} \com{\sigma_y}{\cdot} &&= \begin{bmatrix}
0 & 0 & 0 & 0 \\ 0 & 0 & 0 & \omega \\ 0 & 0 & 0 & 0 \\ 0 & -\omega & 0 & 0
\end{bmatrix}.
\end{alignat}
The Lindbladian is just the sum of these two terms:
\begin{equation}
\dot{\vec{\rho}} = \begin{bmatrix}
0 & 0 & 0 & 0 \\ 0 & -\gamma & 0 & \omega \\ 0 & 0 & -\gamma & 0 \\ 0 & -\omega & 0 & 0
\end{bmatrix} \vec{\rho} = \hat{L} \vec{\rho}
\end{equation}
We take the matrix exponent $\exp(\hat{L} t)$ to get the superoperator $S$:
\begin{equation}
S = \begin{bmatrix}
1 & 0 & 0 & 0 \\
0 & e^{-\frac{\gamma t}{2}}\left( \cosh(\nicefrac{\beta t}{2}) - \frac{\gamma}{\beta} \sinh(\nicefrac{\beta t}{2})\right) & 0 & -\frac{2 \omega}{\beta} e^{-\frac{\gamma t}{2}}\sinh(\nicefrac{\beta t}{2}) \\
0 & 0 & e^{-\frac{\gamma t}{2}} & 0 \\
0 & \frac{2 \omega}{\beta} e^{-\frac{\gamma t}{2}}\sinh(\nicefrac{\beta t}{2}) & 0 & e^{-\frac{\gamma t}{2}}\left( \cosh(\nicefrac{\beta t}{2}) + \frac{\gamma}{\beta} \sinh(\nicefrac{\beta t}{2})\right)
\end{bmatrix}
\end{equation}
where $\beta = \sqrt{\gamma^2 - 4\omega^2}$.

If $\omega$ is large, and $\gamma$ is small (as we hope will be the case in low-noise systems), then $\beta$ will be imaginary, and the hyperbolic functions will become regular trigonometric functions:
\begin{flalign}
\beta &\equiv i \nu \\
\cosh \left(\frac{\beta t}{2} \right) & = \cos \left(\frac{\nu t}{2} \right)\\
\frac{\sinh(\frac{\beta t}{2})}{\beta} & = \frac{\sin(\frac{\nu t}{2})}{\nu} \\
S & \mapsto \begin{bmatrix}
1 & 0 & 0 & 0 \\
0 & e^{-\frac{\gamma t}{2}}\left( \cos(\nicefrac{\nu t}{2}) - \frac{\gamma}{\nu} \sin(\nicefrac{\nu t}{2})\right) & 0 & -\frac{2 \omega}{\nu} e^{-\frac{\gamma t}{2}}\sin(\nicefrac{\nu t}{2}) \\
0 & 0 & e^{-\frac{\gamma t}{2}} & 0 \\
0 & \frac{2 \omega}{\nu} e^{-\frac{\gamma t}{2}}\sin(\nicefrac{\nu t}{2}) & 0 & e^{-\frac{\gamma t}{2}}\left( \cos(\nicefrac{\nu t}{2}) + \frac{\gamma}{\nu} \sin(\nicefrac{\nu t}{2})\right)
\end{bmatrix}
\end{flalign}
We have control over $t$, and we'd like to determine how to set it in order to obtain the maximum-fidelity $Y_{90}$. 
To separate the noise from the gate we'd like to perform, we express the total superoperator as the product of a desired $Y_{90}$ superoperator and a noise operator:
\begin{flalign}
S &= S_{\textrm{Noise}}S_{\textrm{Id}}\\
\therefore S_{\textrm{Noise}} &= S_{\textrm{Id}}^{-1} S \\
S_{\textrm{Id}} &= \begin{bmatrix}
1 & 0 & 0 & 0 \\ 0 & 0 & 0 & 1 \\ 0 & 0 & 1 & 0 \\ 0 & -1 & 0 & 0 
\end{bmatrix} \\ 
\therefore S_{\textrm{Noise}} &= \begin{bmatrix}
1 & 0 & 0 & 0 \\
0 & \frac{2\omega}{\nu} \exp(-\frac{\gamma t}{2}) \sin(\frac{\nu t}{2}) & 0 & \exp(-\frac{\gamma t}{2}) \left( \frac{\gamma}{\nu} \sin(\frac{\nu t}{2}) -\cos (\frac{\nu t}{2})\right) \\
0 & 0 & \exp(-\gamma t) & 0 \\
0 & \exp(-\frac{\gamma t}{2}) \left( \frac{\gamma}{\nu} \sin(\frac{\nu t}{2}) + \cos (\frac{\nu t}{2})\right) & 0 & \frac{2\omega}{\nu} \exp(-\frac{\gamma t}{2}) \sin(\frac{\nu t}{2})
\end{bmatrix} 
\end{flalign}
To find the channel fidelity $F_{\Lambda} = \elem{\Omega}{\Lambda \otimes \id \left( \proj{\Omega} \right)}{\Omega}$ (where $\ket{\Omega}$ is a Bell state), we take the trace of this superoperator and divide by 4 (I won't prove this here, but leave it as an exercise):
\begin{equation}
F_{S_{\textrm{Noise}}} = \frac{1}{4} \left[ 1 + \frac{4\omega}{\nu} \exp\left(-\frac{\gamma t}{2}\right) \sin\left(\frac{\nu t}{2}\right) + \exp\left(-\gamma t\right) \right]
\end{equation}
To find out how long to leave the Hamiltonian on, we try to optimize this fidelity over $t$:
\begin{flalign}
\dfrac{dF_{S_{\textrm{Noise}}}}{dt} &= \frac{1}{2} \, \omega \cos\left(\frac{1}{2} \, \nu t\right) e^{\left(-\frac{1}{2} \, \gamma t\right)} - \frac{\gamma \omega e^{\left(-\frac{1}{2} \, \gamma t\right)} \sin\left(\frac{1}{2} \, \nu t\right)}{2 \, \nu} - \frac{1}{4} \, \gamma e^{\left(-\gamma t\right)} = 0 \\ 
\therefore \sin\left(\frac{1}{2} \, \nu t\right) &= \frac{{\left(2 \, \nu \omega \cos\left(\frac{1}{2} \, \nu t\right) e^{\left(\gamma t\right)} - \gamma \nu e^{\left(\frac{1}{2} \, \gamma t\right)}\right)} e^{\left(-\gamma t\right)}}{2 \, \gamma \omega}
\end{flalign}
This formula is not analytically soluble, so we just set $t=\frac{\pi}{\nu}$, to get the following (ridiculously good-looking) superoperator:
\begin{flalign}
\begin{bmatrix}
1 & 0 & 0 & 0 \\
0 & -\frac{\gamma e^{\left(-\frac{\pi \gamma}{2 \, \nu}\right)}}{\nu} & 0 & \frac{2 \, \omega e^{\left(-\frac{\pi \gamma}{2 \, \nu}\right)}}{\nu} \\
0 & 0 & e^{\left(-\frac{\pi \gamma}{\nu}\right)} & 0 \\
0 & -\frac{2 \, \omega e^{\left(-\frac{\pi \gamma}{2 \, \nu}\right)}}{\nu} & 0 & \frac{\gamma e^{\left(-\frac{\pi \gamma}{2 \, \nu}\right)}}{\nu}.
\end{bmatrix}
\end{flalign}
We subtract off a term $2\frac{\omega}{\nu}\exp\left( -\frac{\pi \gamma}{2 \nu} \right) Y_{90}$ to obtain the remaining diagonal part of the superoperator:
\begin{flalign}
\begin{bmatrix}
-\frac{2 \, \omega e^{\left(-\frac{\pi \gamma}{2 \, \nu}\right)}}{\nu} + 1 & 0 & 0 & 0 \\
0 & -\frac{\gamma e^{\left(-\frac{\pi \gamma}{2 \, \nu}\right)}}{\nu} & 0 & 0 \\
0 & 0 & -\frac{2 \, \omega e^{\left(-\frac{\pi \gamma}{2 \, \nu}\right)}}{\nu} + e^{\left(-\frac{\pi \gamma}{\nu}\right)} & 0 \\
0 & 0 & 0 & \frac{\gamma e^{\left(-\frac{\pi \gamma}{2 \, \nu}\right)}}{\nu}
\end{bmatrix}
\end{flalign}
This is equivalent to a Pauli map with the following probabilities:
\begin{flalign}
&p_I = -\frac{\omega e^{\left(-\frac{\pi \gamma}{2 \, \nu}\right)}}{\nu} + \frac{1}{4} \, e^{\left(-\frac{\pi \gamma}{\nu}\right)} + \frac{1}{4} \quad
p_X = -\frac{\gamma e^{\left(-\frac{\pi \gamma}{2 \, \nu}\right)}}{2 \, \nu} - \frac{1}{4} \, e^{\left(-\frac{\pi \gamma}{\nu}\right)} + \frac{1}{4} \nonumber \\
&p_Y = -\frac{\omega e^{\left(-\frac{\pi \gamma}{2 \, \nu}\right)}}{\nu} + \frac{1}{4} \, e^{\left(-\frac{\pi \gamma}{\nu}\right)} + \frac{1}{4} \quad
p_Z = \frac{\gamma e^{\left(-\frac{\pi \gamma}{2 \, \nu}\right)}}{2 \, \nu} - \frac{1}{4} \, e^{\left(-\frac{\pi \gamma}{\nu}\right)} + \frac{1}{4} 
\end{flalign}
These probabilities approach $0$ in the small $\gamma$ limit and approach $\frac{1}{4}$ in the large $\gamma$ limit. 
I'm willing to bet that they are always between $0$ and $1$.
\section{Amplitude Damping}
We'd also like, if possible, to consider the effect of amplitude damping (or $T_1$ noise) on the $Y_{90}$ gate. 
It will likely not be possible to express this noise as a mixed-Clifford channel, given that it is non-unital.
There are two approaches to coping with this; either we obtain a mixed-Clifford channel which optimally approximates the gate, or we fold the noisy $Y_{90}$ into the nearest state-preparation or measurement location, modeling the effect of $T_1$ on a mixture of $\proj{0}$ and $\proj{1}$ rather than superpositions. 
This second approach ought to be easier (analytically soluble, rather than requiring an SDP), and apply most of the time (since syndrome extraction from a CSS code usually only uses $H$/$Y_{90}$ gates to prepare superposition states and rotate measurement bases), so I'll try that first. 

We calculate the effect of a $\sigma_-$ dissipator in the Pauli basis:
\begin{flalign}
\diss{\sigma_-}{\id} &= \com{\sigma_-}{\sigma_+} = \sigma_z \\ 
\diss{\sigma_-}{\sigma_x} &= \lindthree{\sigma_-}{\sigma_x}{\sigma_+} = \ketbra{0}{1} \sigma_x \ketbra{1}{0} - \dfrac{1}{2} \acom{\proj{1}}{\sigma_x} = -\dfrac{1}{2} \sigma_x \\
\diss{\sigma_-}{\sigma_y} &= -\dfrac{1}{2}\sigma_y \\ 
\diss{\sigma_-}{\sigma_z} &= \ketbra{0}{1} \sigma_z \ketbra{1}{0} - \dfrac{1}{2} \acom{\proj{1}}{\sigma_z} = - \sigma_z
\end{flalign}
This allows us to express the new Lindbladian in matrix form (note that I mess with coefficient definitions to try to get the matrix to look nice, your coefficients may vary):
\begin{flalign}
\dot{\rho} &= -i\frac{\omega}{2} \com{\sigma_y}{\rho} + \frac{\gamma_{\phi}}{2} \diss{\sigma_z}{\rho} + \gamma_{-} \diss{\sigma_-}{\rho} \\ 
\dot{\vec{\rho}} &= \begin{bmatrix}
0 & 0 & 0 & -\gamma_{-} \\ 0 & -\gamma_{\phi} - \frac{1}{2}\gamma_{-} & 0 & \omega \\ 0 & 0 & -\gamma_{\phi}-\frac{1}{2}\gamma_{-} & 0 \\ \gamma_{-} & -\omega & 0 & 0
\end{bmatrix} \vec{\rho} = \hat{L} \vec{\rho}
\end{flalign}
\section{Questions}
\begin{enumerate}
\item Show that ``shorting'' the gate time optimally (maximizing the channel fidelity) doesn't appreciably raise the fidelity over just setting $\omega t = \frac{\pi}{\nu}$.
\item How does all this change when we add amplitude damping? 
\end{enumerate}

\end{document}